%
% DO NOT CHANGE THIS TEMPLATE!
%
% To allow combining the lecture notes from different scribes in a
% simple way all scribes must use the same header.
%
% If you think the template lack a package, then contact the lecturer,
% and we will probably extend the template with your favorite package.
%

\documentclass[11pt,a4paper,twoside]{article}
\usepackage{alg}
\usepackage[T1]{fontenc}
\usepackage{graphicx}
\usepackage{subfigure}
\usepackage{amssymb}
\usepackage{amsmath}

\begin{document}
\pagestyle{myheadings}
\thispagestyle{plain}

\newcommand{\lecture}[5] { %
\newcommand{\fdatum}{#1}%
\newcommand{\fnummer}{#2}%
\newcommand{\frubrik}{#3}%
\newcommand{\fforelasare}{#4}%
\newcommand{\fskribent}{#5}%
\markboth{DD2458 -- Popup VT 2011}{\frubrik}%
\hrule
\begin{center}
\large \bf 
DD2458, Problem Solving and Programming Under Pressure

\vspace*{2mm}
\Large \bf 
Lecture \fnummer: \frubrik
\end{center}
\small
\noindent Date: \fdatum
\\
Scribe(s): \fskribent
\\
Lecturer: \fforelasare
\\
\hrule
\vspace{5mm}
}


\newtheorem{theorem}{Theorem}[section]
\newtheorem{lemma}[theorem]{Lemma}
\newtheorem{corollary}[theorem]{Corollary}
\newtheorem{claim}[theorem]{Claim}
\newtheorem{definition}[theorem]{Definition}

% \lecture{date}{lecture number}{lecture headline}{lecturer}{scribe}
%
%
%

%%%%% Edit after this row. The above should not need editing. %%%%%

%%%%% Define your own stuff here. %%%%%

\newcommand{\zed}{\mathbb{Z}}
\newcommand{\nat}{\mathbb{N}}
\newcommand{\GF}[1]{\mathrm{GF}_{#1}}

%%%%%%%%%%%%%%%%%%%%%%%%%%%%%%%%%%%%%%%


\lecture{2011-02-01}{3}{The Hitchhiker's Guide to Debugging and Testing}{Alexander Baltatzis}{Joel Bohman, Linus Wallgren, Oskar Werkelin Ahlin}

\noindent
Debugging is the art of finding bugs in code and to be able to get rid of them.
Nowadays, this art has become a serious profession for many programmers, who
spend their working days hunting bugs. The first step in learning about
debugging is to know about one of the most well-know bugs in history. In 1946,
at Harvard University, a moth caused a system failure by being trapped inside a
relay in one of the electromechanical computers used at the time. Even though
it is not common nowadays that program bugs are caused by actual insects, their
source is still often evasive and hard to detect. This is a short guide for
anyone interested in learning more about debugging and testing.

%%%%%%%%%%%%%%%%%%%%%%%%%%%%%%%%%%%%%%%%%%%%%%%%%%%%%%%%%%%%%%%%%%%%%%%%%%%%%%%%
%%%%%%%%%%%%%%%%%%%%%%%%%%%%%%%%%%%%%%%%%%%%%%%%%%%%%%%%%%%%%%%%%%%%%%%%%%%%%%%%
%%%%%%%%%%%%%%%%%%%%%%%%%%%%%%%%%%%%%%%%%%%%%%%%%%%%%%%%%%%%%%%%%%%%%%%%%%%%%%%%

\section{Testing}

Testing the code you have written is of great importance. It is the way you
confirm that your program runs the way you expect it to. Having a well tested
program means that you can be sure the program will not behave unexpectedly.
There are a number of testing techniques available and the most important ones
are listed below.


\subsection{Documenting tests using a test matrix}

You should always document your tests, in order to be able to look back and see
what you did and did not test before, and what happened if you ran the
program in that specific way. A test matrix is a good way to easily implement the
fundamental ideas of documented testing. It contains a collection of tests, and
for each test we list four points:

\begin{itemize}
    \item Prerequisites - what are the conditions for running this test?
    \item The test itself - what part of the program does it execute?
    \item The expected result - what do we expect to achieve?
    \item The results from the program - what result did we get?
\end{itemize}

\noindent{Following} this principle, a matrix is formed as we fill it up with
rows, an example of a test matrix can be found in Table 1.

\begin{center}
\begin{table}[h!]
\begin{tabular}{|l|p{2.5cm}|p{2cm}|p{2.5cm}|p{2.5cm}|}
\hline
Test \# & Prerequisites & The test & Expected result & Test results \\ \hline 1
& Compiled with g++ -O2 -g on an Intel x86-64 architecture, debian stable. &
Provoke the program to divide by zero. & An error mentioning divide by zero. &
The system crashed and burned. \\
\hline
\end{tabular}
\caption{An example of a test matrix.}
\end{table}
\end{center}

\noindent{This} is a comprehensive way to document your testing. As the matrix fills up,
more knowledge is gained about the program's functionality, and actions can be
taken thereafter. Documenting your tests like this is strongly
advisable.
 

\subsection{State-based testing}

State-based testing is a way of testing the program using abstract states and state
transitions. First, define different states in the program. Then define
transitions between those states. Transitions are caused by program events. The
goal is to confirm that the program changes state properly. Given a starting
state (e.g. specific program input), we want to make sure that the transition
(e.g. a calculation) takes us to the correct final state (e.g. outputting the
correct result). A state is often closely related to the current values of the
different program variables.
 

\subsubsection{Boundary values}

Boundary value errors and off-by-one errors are errors that occur at the edges
of input. The program might run as expected for most input, but when given
certain values, a bug occurs. If your program should accept input in a certain
range, it is usually a good idea to test both edges of that range. There are a
couple of common boundary values that one should test the program for, to
confirm that it does what is expected from it.

\begin{itemize}
    \item Lower bounds:
    \begin{itemize}
        \item 0, -1 and 1.
        \item The empty string, ""
        \item Empty list, []
        \item NULL or equivalent.
        \item Signed 32-bit integer: $-(2^{31}-1)$, $-2^{31}$.
        \item Signed 64-bit integer: $-(2^{63}-1)$, $-2^{63}$.
    \end{itemize}
    \item Upper bounds:
    \begin {itemize}
        \item Signed 32-bit integer: $2^{31}-1$, $2^{31}$.
        \item Unsigned 32-bit integer: $2^{32}-1$, $2^{32}$.
        \item Signed 64-bit integer: $2^{63}-1$, $2^{63}$.
        \item Unsigned 64-bit integer: $2^{64}-1$, $2^{64}$.
        \item A string of maximum accepted length.
        \item A list of maximum accepted size.
    \end{itemize}
\end{itemize}

The reader should note that an error can occur from a combination of these
boundary values, not only when a single variable in a state assume a boundary
value.


\subsubsection{Program behaviour}

It is always preferred to have a program with a well defined behaviour, even
though the input is invalid. Having a program with undefined behaviour can be
risky and should be avoided. Testing this can be done by using random values as
input, which is likely to violate the original program specifications.

You are advised to check all code you have written and make sure it works. Even
though a part of the program is not currently used, it might be in the future.
For example, it is a good idea to have a working copy-constructor for future
use. The copy-constructor is a good example of a place where unforeseen
side-effects might arise. Such things should be tested, because even though
those side-effects seem harmless, the program might crash once the side-effect
arises in combination with another program functionality.


\subsection{Behavioural testing}

Behavioural testing is a more general testing method than state-based testing.
Instead of looking at states in code, behavioural testing focuses on the
functionally of a chunk of code, ignoring the internal states. An example could
be a breadth-first search, in behavioural testing we look at the result of the
breadth-first search compared to program input, and not the internal states of
each stage of the search. Behavioural testing is vital when developing
graphical user interfaces, to make sure that internal states match what is
displayed on the screen.


\subsection{Testing manually}

Manual testing is a widely employed technique and it is almost guaranteed to be
used by any developer. When doing manual testing the developer or another
person takes the role of the end user and gives the program input to see that
it produces an acceptable result. This is related to behavioural testing but
instead of looking at the code you only look at the program and the program
functionality.

Manual testing is almost always a bad idea if the developer is the only person
doing it. A developer most certainly has preconceptions of his or her code
which might lead to not all functionality is equal tested. Manual testing is
also hard to reproduce, to help with that a test script could be written to
tell the person acting as the end user to test different functions of the
program. But the biggest problem with manual testing is regression. If a manual
test find a bug and the developer fix the bug, how can he be sure that the fix
did not break earlier fixes or introduce a new bug? It is a big problem if
only manual testing is employed because old tests are not saved.

\subsection{Automated testing}

Running tests by hand can be a tiresome activity and usually also bad practice.
Therefore you would want to automate it. If you have input data and reference
output data you could compare the program output with the reference output
data. Comparing the files can be done automatically, for example using diff:

\begin{quote}
\begin{verbatim}
$ program < input > result
$ diff result reference.output 
\end{verbatim}
\end{quote}

\noindent{Remember} that it is always a good idea to save all tests you have previously
performed. As the test suite grows, a script or similar automatic mechanism
should be used to run the test harness. 

\subsubsection{Unit testing}

Unit testing is a type of testing that is always automated. The goal of unit
testing is to test every unit of code. A unit is a measure of the smallest part
of code that can be tested in your program. Libraries for unit testing are
available in almost any programming language and on almost any platform.

%%%%%%%%%%%%%%%%%%%%%%%%%%%%%%%%%%%%%%%%%%%%%%%%%%%%%%%%%%%%%%%%%%%%%%%%%%%%%%%%
%%%%%%%%%%%%%%%%%%%%%%%%%%%%%%%%%%%%%%%%%%%%%%%%%%%%%%%%%%%%%%%%%%%%%%%%%%%%%%%%
%%%%%%%%%%%%%%%%%%%%%%%%%%%%%%%%%%%%%%%%%%%%%%%%%%%%%%%%%%%%%%%%%%%%%%%%%%%%%%%%

\section{When an error occurs}

\subsection{Errors}

\subsubsection{Compile errors}

This is an error that you receive when you compile your code. The compiler
stops and tells you there is an error that must be fixed before it can be
compiled. Compile time errors are the most preferable errors, as you notice
them right away and can correct them fast. In some programming languages, some
logical errors will transform into compile errors, due to strong typing in that
particular language. 

\subsubsection{Compile warnings}

Compile time warnings generally do not stop the compiling process, but they are
usually good to take care of. It might be that you get warnings which are
unavoidable, for example warnings about deprecated methods. If a large amount
of these are generated at compile time, more serious warnings might be
overlooked, because of the sheer amount of warnings. If you get warnings that
you know will not do any harm, it is advisable to use compiler flags to hide
these. In gcc, this can be done using the -W flag.

\subsubsection{Run time errors}

A run time error occurs after compilation and during the execution of the
program. Run time errors are programming errors not detectable by the compiler,
making them harder to find and remove. An example of a typical run time error
is when you try to access an array element outside of the array boundaries or
when you try to follow a null pointer.

\subsubsection{Logical errors}

Logical errors are the most difficult kinds of errors. They are especially
difficult to detect as the program does nothing wrong, it does as it was
instructed, it is just not what you want it to do. The error is only in the
behaviour of the program. As these are the most difficult errors to handle we
need to employ a couple of debugging techniques.

\subsection{Debugging}

Debugging is the art of localizing a bug and then finding the fix. Debugging is
suitable for both run time errors and logical errors. 

\subsubsection{Compiler debug flag}

For some compilers, for example gcc and g++\footnote{http://gcc.gnu.org}, the
debugging flag changes the memory structure of the program. When the debug flag
is active the compiler adds padding around variables and arrays in order to
find out if something goes wrong during execution. By changing the memory
structure you might alter the behaviour of the program, as it is gentler to
overflows and similar errors.  Even if the program runs fine with the debug
flag it might crash without it.  Another problem with the debug flag is that it
makes it easier for others to reverse-engineer the program. The reason is that
the debug flag incorporates more information about the program itself into the
binary. Because of these difficulties it is important to test both with and
without the debug flag active.


\subsubsection{Trace output}

Printing information about the state of the program to the screen is a
powerful, yet simple way of debugging. The concept is usually referred to as
trace outputs. A trace output should always contain two parts to be useful,
location and state. Location should clearly and unambiguously specify where in
the program the output is generated. State describes the current state of the
program at that location. A state is almost always described in terms of
variable names and values. An example of trace output:

\begin{quote}
\begin{verbatim}
foo(): bar=5, a=10
\end{verbatim}
\end{quote}

\noindent{The} example could be be achieved with for instance printf, which is available
in most languages. In C it might look like this:

\begin{quote}
\begin{verbatim}
printf("foo(): bar=%d, a=%d\n", bar, a);
\end{verbatim}
\end{quote}

\noindent{And} it is very similar in Java:

\begin{quote}
\begin{verbatim}
System.out.printf("foo(): bar=%d, a=%d\n", bar, a);
\end{verbatim}
\end{quote}


\subsubsection{Binary search the code}
If the program is divided into specific parts, and the program has an error,
 you can find out where the error is located by running different parts of the 
 program at a time. Start by commenting out half of the program, if the 
 error is gone remove half of those comments and see if it is still gone. Keep
 doing this until you have located the error successfully. This is often an efficient
 way of locating errors.

\subsubsection{Logging}

Logging is an extension of trace output and is a good way of handling a lot of
output. Especially daemons and other programs with no terminal attached benefit
from the use of logging to communicate their output. Usually logging happens at
several levels, for example \textsc{INFO}, \textsc{WARNING} and \textsc{ERROR}.
The logging levels are supposed to describe the output generated. It might be
so that it is only interesting to know about errors in the code, in which case
the \textsc{ERROR} level might suffice. If you would like to trace the program
execution you might want more verbose output, in which case \textsc{INFO}
probably is the right choice. 

It is also common practice to direct logging output to files, in order to
review them later on. This also helps if there is a lot of output, as it would
be to much to go through live as the execution happens.

\subsection{Debugging tools}

There are many tools that you can use in order to master the art of debugging.
If you are a command-line oriented person your tool belt should at the very
least contain the debugger's gdb \footnote{http://www.gnu.org/software/gdb/}
and valgrind \footnote{http://valgrind.org/}, a source editor like vim and the
GNU compiler family. If you are more of a graphical user interface person you
could use a full-fledged integrated development environment (IDE). An IDE is
basically a combination of a source editor, debugger and compiler. A couple of
examples includes:

\begin{itemize}
\item Visual Studio - Windows
\item Eclipse - Cross-platform
\item XCode - Mac OS X
\item Netbeans - Cross-platform
\item Codeblocks - Cross-platform
\end{itemize}


\noindent{It} should be mentioned that it is always a good idea to know the
command-line tools as you can not always be sure to have access to a graphical
user interface when you are debugging.  

\subsubsection{Debugging using breakpoints}

Most IDEs support debugging using so called breakpoints. A breakpoint is a mark
that you put in your code, telling the IDE to halt execution when it gets to
that line of code. Then, you can calmly step through each line of code after
that, and see how the program state changes throughout the execution of the
program.

%%%%%%%%%%%%%%%%%%%%%%%%%%%%%%%%%%%%%%%%%%%%%%%%%%%%%%%%%%%%%%%%%%%%%%%%%%%%%%%%
%%%%%%%%%%%%%%%%%%%%%%%%%%%%%%%%%%%%%%%%%%%%%%%%%%%%%%%%%%%%%%%%%%%%%%%%%%%%%%%%
%%%%%%%%%%%%%%%%%%%%%%%%%%%%%%%%%%%%%%%%%%%%%%%%%%%%%%%%%%%%%%%%%%%%%%%%%%%%%%%%

\section{Error prevention}

\subsection{Test-driven development}

The idea behind test-driven programming is that you write the tests before you
begin working on the implementation, that way you know what to expect from your
program. Test-driven programming also gives you a clear specification of
what the program is supposed to do, which most likely will make development
much more comprehensive.


\subsection{Contract programming}

Contract programing has a simple core idea, make contracts in the program. In
other words, make sure that certain conditions are met throughout the execution
of the program.

The most strict way of achieving this is to stop the program if any of the
conditions are not met. This could be done with assert (in C/C++) and
equivalents in most programming languages, for example:

\begin{quote}
\begin{verbatim}
assert(x == 4711);
\end{verbatim}
\end{quote}

\noindent{Another} 
less destructive way to handle unexpected behaviour is with exceptions.
Exceptions are thrown when unwanted behaviour appears. This is a way of
signaling that an error has occurred, but letting some other part of the
program handle it and determine what to do. It might be that we want to halt
the execution, as with assert, but we might also only want to log it, or
completely ignore it. This way, exceptions give us much more freedom than using
asserts.

\subsection{Code structure}
Always keep your code structured. This does not only reduce the chances of
introducing bugs, it also makes the debugging job much easier once a bug has
been introduced. It can reduce the amount of work laid down by more than you
usually expect.

%%%%%%%%%%%%%%%%%%%%%%%%%%%%%%%%%%%%%%%%%%%%%%%%%%%%%%%%%%%%%%%%%%%%%%%%%%%%%%%%
%%%%%%%%%%%%%%%%%%%%%%%%%%%%%%%%%%%%%%%%%%%%%%%%%%%%%%%%%%%%%%%%%%%%%%%%%%%%%%%%
%%%%%%%%%%%%%%%%%%%%%%%%%%%%%%%%%%%%%%%%%%%%%%%%%%%%%%%%%%%%%%%%%%%%%%%%%%%%%%%%

\section{Profiling your code}
You have got your program running and giving correct output, but it is not fast
enough. In order to be able to know which parts of the program that need to be
optimized, we usually use a profiler.  A profiler tracks the performance in
different program parts during execution, and displays the result of this in a
table after finished execution. This shows the programmer which methods or
functions are time bottle necks, and need to be optimized.

\subsection{Popular profilers}
There is many good profilers available, but here we list the most popular ones.
\begin{itemize}
    \item Gprof - the GNU Profiler. A popular profiler among Unix developer.
    Works for any language supported by GCC (the GNU Compiler Collection).

    \item Valgrind - A usefull tool for detecting memory leaks, but can also be
    used for general memory debugging as well as profiling.

    \item java -Xprof - Java's built-in profiler. Not very extensive, but very
    easy to use.  Simply run the java program with the -Xprof flag.

\end{itemize}
%%%%%%%%%%%%%%%%%%%%%%%%%%%%%%%%%%%%%%%%%%%%%%%%%%%%%%%%%%%%%%%%%%%%%%%%%%%%%%%%
%%%%%%%%%%%%%%%%%%%%%%%%%%%%%%%%%%%%%%%%%%%%%%%%%%%%%%%%%%%%%%%%%%%%%%%%%%%%%%%%
%%%%%%%%%%%%%%%%%%%%%%%%%%%%%%%%%%%%%%%%%%%%%%%%%%%%%%%%%%%%%%%%%%%%%%%%%%%%%%%%

\section{If you get stuck}

If you managed to read this far you are now most certainly ready start your
quest to perfect the art of debugging. Because hunting bugs can be a tiresome
activity that can make the most sane person insane, you need a couple of tips
to carry with you. If you find yourself in the situation where you are stuck
and all hope seems gone, try one of the following: 

\begin{itemize} 
    \item Do something else! A fresh mind and body can do miracles. Take a walk.
          Have something to eat. Sleep on it. 
    \item Try to explain the problem to another person. By explaining your 
          problem in words it will get you to think about the problem and you 
          might see it from a new perspective.  
    \begin{itemize}
        \item If this does not help, show that person or any other person
              your code. Another pair of eyes can see something you might have 
              missed. This is called peer-reviewing.
    \end{itemize}
    \item Use pen and paper. Draw some pictures describing the program flow or 
          how the program is run given certain input. Compare it to the actual
          execution. This might give you another view of the problem and how it 
          should be solved.
    \item Start from the beginning. Understanding the logic to solve an 
          assignment is what a programmer spends most time doing. When you have
          the logic down you can write code at a higher pace. Chances are you
          will not hit the same problem the second time around. Refactoring your
          code is often faster than trying to figure out what was wrong with 
          your first version.
\end{itemize}

\end{document}
