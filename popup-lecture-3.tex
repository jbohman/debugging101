%
% DO NOT CHANGE THIS TEMPLATE!
%
% To allow combining the lecture notes from different scribes in a
% simple way all scribes must use the same header.
%
% If you think the template lack a package, then contact the lecturer,
% and we will probably extend the template with your favorite package.
%

\documentclass[11pt,a4paper,twoside]{article}
\usepackage{alg}
\usepackage[T1]{fontenc}
\usepackage{graphicx}
\usepackage{subfigure}
\usepackage{amssymb}
\usepackage{amsmath}

\begin{document}
\pagestyle{myheadings}
\thispagestyle{plain}

\newcommand{\lecture}[5] { %
\newcommand{\fdatum}{#1}%
\newcommand{\fnummer}{#2}%
\newcommand{\frubrik}{#3}%
\newcommand{\fforelasare}{#4}%
\newcommand{\fskribent}{#5}%
\markboth{DD2458 -- Popup VT 2011}{\frubrik}%
\hrule
\begin{center}
\large \bf 
DD2458, Problem Solving and Programming Under Pressure

\vspace*{2mm}
\Large \bf 
Lecture \fnummer: \frubrik
\end{center}
\small
\noindent Date: \fdatum
\\
Scribe(s): \fskribent
\\
Lecturer: \fforelasare
\\
\hrule
\vspace{5mm}
}


\newtheorem{theorem}{Theorem}[section]
\newtheorem{lemma}[theorem]{Lemma}
\newtheorem{corollary}[theorem]{Corollary}
\newtheorem{claim}[theorem]{Claim}
\newtheorem{definition}[theorem]{Definition}

% \lecture{date}{lecture number}{lecture headline}{lecturer}{scribe}
%
%
%

%%%%% Edit after this row. The above should not need editing. %%%%%

%%%%% Define your own stuff here. %%%%%

\newcommand{\zed}{\mathbb{Z}}
\newcommand{\nat}{\mathbb{N}}
\newcommand{\GF}[1]{\mathrm{GF}_{#1}}

%%%%%%%%%%%%%%%%%%%%%%%%%%%%%%%%%%%%%%%


\lecture{2011-02-01}{3}{The Hitchhiker's Guide to Debugging and Testing}{Alexander Baltatzis}{Joel Bohman, Linus Wallgren, Oskar Werkelin Ahlin}

\noindent
Debugging is the art of finding bugs in code and to be able to get rid of them.
Nowadays, this art has become a serious profession for many programmers, who
spend their working days hunting bugs. The first step in learning about
debugging is to know about one of the most well-know bugs in history. In 1946,
at Harvard University, a moth caused a system failure by being trapped inside a
relay in one of the electromechanical computers used at the time. Even though
it is not common nowadays that program bugs are caused by actual insects, their
source is still often evasive and hard to detect. This is a short guide for
anyone interested in learning more about debugging and testing.

%%%%%%%%%%%%%%%%%%%%%%%%%%%%%%%%%%%%%%%%%%%%%%%%%%%%%%%%%%%%%%%%%%%%%%%%%%%%%%%%
%%%%%%%%%%%%%%%%%%%%%%%%%%%%%%%%%%%%%%%%%%%%%%%%%%%%%%%%%%%%%%%%%%%%%%%%%%%%%%%%
%%%%%%%%%%%%%%%%%%%%%%%%%%%%%%%%%%%%%%%%%%%%%%%%%%%%%%%%%%%%%%%%%%%%%%%%%%%%%%%%

\section{Testing}

Testing the code you have written is of great importance. It is the way you
confirm that your program runs the way you expect it to. Having a well tested
program means that you can be sure the program will not behave unexpectedly.
There are a number of testing techniques available and the most important ones
are listed below.


\subsection{Documenting tests using a test matrix}

\begin{center}
\begin{tabular}{|l|p{2.5cm}|p{2cm}|p{2.5cm}|p{2.5cm}|}
\hline
Test \# & Prerequisites & The test & Expected result & Test results \\ \hline 1
& Compiled with g++ -O2 -g on an Intel x86-64 architecture, debian stable. &
Provoke the program to divide by zero. & An error mentioning divide by zero. &
The system crashed and burned. \\
\hline
\end{tabular}
\end{center}

\subsection{State-based testing}

\subsubsection{Boundary values}

\subsubsection{Well defined behaviour}

\subsection{Behavioural testing}

\subsection{Testing manually}

\subsection{Automated testing}

\subsubsection{Unit testing}

%%%%%%%%%%%%%%%%%%%%%%%%%%%%%%%%%%%%%%%%%%%%%%%%%%%%%%%%%%%%%%%%%%%%%%%%%%%%%%%%
%%%%%%%%%%%%%%%%%%%%%%%%%%%%%%%%%%%%%%%%%%%%%%%%%%%%%%%%%%%%%%%%%%%%%%%%%%%%%%%%
%%%%%%%%%%%%%%%%%%%%%%%%%%%%%%%%%%%%%%%%%%%%%%%%%%%%%%%%%%%%%%%%%%%%%%%%%%%%%%%%

\section{When an error occurs}

\subsection{Errors}

\subsubsection{Run time errors}

\subsubsection{Logical errors}

\subsubsection{Compile errors}

\subsubsection{Warnings}

\subsection{Debugging}

\subsubsection{Trace output}

\subsubsection{Logging}

\subsection{Tools}


%%%%%%%%%%%%%%%%%%%%%%%%%%%%%%%%%%%%%%%%%%%%%%%%%%%%%%%%%%%%%%%%%%%%%%%%%%%%%%%%
%%%%%%%%%%%%%%%%%%%%%%%%%%%%%%%%%%%%%%%%%%%%%%%%%%%%%%%%%%%%%%%%%%%%%%%%%%%%%%%%
%%%%%%%%%%%%%%%%%%%%%%%%%%%%%%%%%%%%%%%%%%%%%%%%%%%%%%%%%%%%%%%%%%%%%%%%%%%%%%%%

\section{Models}

\subsection{Test-driven}

\subsection{Contract programming}

%%%%%%%%%%%%%%%%%%%%%%%%%%%%%%%%%%%%%%%%%%%%%%%%%%%%%%%%%%%%%%%%%%%%%%%%%%%%%%%%
%%%%%%%%%%%%%%%%%%%%%%%%%%%%%%%%%%%%%%%%%%%%%%%%%%%%%%%%%%%%%%%%%%%%%%%%%%%%%%%%
%%%%%%%%%%%%%%%%%%%%%%%%%%%%%%%%%%%%%%%%%%%%%%%%%%%%%%%%%%%%%%%%%%%%%%%%%%%%%%%%

\section{If you get stuck}

\end{document}
